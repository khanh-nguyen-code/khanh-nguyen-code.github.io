\documentclass{article}
\usepackage[utf8]{inputenc}

\usepackage{natbib}
\bibliographystyle{plain}


\newtheorem{theorem}{Theorem}
\newtheorem{corollary}{Corollary}
\newtheorem{lemma}{Lemma}

\title{paxos-algorithm}
\author{Khanh Nguyen}
\date{January 2022}

\begin{document}

\maketitle

\section{The Paxos Algorithm}

    In Paxos algorithm, there are two type of machines: $proposer$ and $acceptor$ where each $proposer$ holds a $value$ and the goal of Paxos algorithm is to reach a consensus in the set of $acceptor$ on a single $value$.

    Let assign a partition of natural numbers to each $proposer$ (each partition is a disjoint subset of natural numbers).
    
    The algorithm below describes the simple version of Paxos algorithm \cite{lamport2001paxos}
    
    \subsection{$proposer$}
    \label{algorithm:proposer}
    
    \subsubsection{phase 1: prepare}
    \label{algorithm:proposer:prepare}
    
        \fbox{\begin{minipage}{30em}
    
        \textbf{input:} a $value$ the $proposer$ intended to propose
    
        \textbf{1} choose a number from its partition, namely $proposal\_number$
    
        \textbf{2} broadcast the prepare message to all $acceptor$
    
            \begin{center}
                PREPARE\_REQUEST \{$proposal\_number$\}
            \end{center}
    
        \textbf{3} if it receives responses from a majority of $acceptor$, do \textbf{phase 2}
    
        \end{minipage}}
    
    \subsubsection{phase 2: accept}
    \label{algorithm:proposer:accept}
    
        \fbox{\begin{minipage}{30em}
    
        \textbf{4} receives 

            \begin{center}
                PREPARE\_RESPONSE \{\{$proposal\_number$, $value$\} or $Null$\}
            \end{center}

        \textbf{5} if all PREPARE\_RESPONSE contain $Null$, pick the input $value$ the $proposer$ intended to propose. Else, pick the $value$ corresponding to the highest $proposer\_number$.
        
        \textbf{6} broadcast the accept message to all $acceptor$
            
            \begin{center}
                ACCEPT\_REQUEST \{$proposal\_number$, $value$\}
            \end{center}
            
        \end{minipage}}

    \subsection{$acceptor$}
    \label{algorithm:acceptor}
    
    an $acceptor$ holds an accepted proposal \{$proposal\_number$, $value$\} in its stable storage.
    
    \subsubsection{on receiving prepare request}
    \label{algorithm:acceptor:prepare}
    
        \fbox{\begin{minipage}{30em}
    
        \textbf{input:} PREPARE\_REQUEST \{$proposal\_number$\}
    
        \textbf{1} if the accepted proposal is not set or $proposal\_number$ is greater than the accepted proposal, reply with the accepted proposal, namely $promise$
        
            \begin{center}
                PREPARE\_RESPONSE \{\{$proposal\_number$, $value$\} or $Null$\}
            \end{center}
            
        \end{minipage}}
    
    \subsubsection{on receiving accept request}
    \label{algorithm:acceptor:accept}
    
        \fbox{\begin{minipage}{30em}
    
        \textbf{input:} ACCEPT\_REQUEST \{$proposal\_number$, $value$\}
    
        \textbf{1} if the accepted proposal is not set or $proposal\_number$ is greater than the accepted proposal, set the accepted proposal to the input, namely $accept$
            
        \end{minipage}}


\section{The Proof of Correctness}

    If the proposal \{$m$, $u$\} is accepted by majority of $acceptor$, the algorithm reaches the consensus on value $u$.
    
    \begin{theorem}
    \label{theorem:1}
    If the proposal \{$m$, $u$\} is accepted by majority of $acceptor$, any issued proposal \{$n$, $v$\} with $m < n$ has $v = u$
    \end{theorem}
    
    Since all proposals accepted by majority of $acceptor$ must be issued at some point, there cannot be any two proposals accepted by majority of $acceptor$ with two different values. A straight forward corollary of theorem \ref{theorem:1}
    
    \begin{corollary}
    \label{corollary:1}
    All proposals accepted by majority of $acceptor$ has the same value.
    \end{corollary}
    
    By guaranteeing corollary \ref{corollary:1}, $acceptor$ can notify the consensus to $proposer$ by replying if they accept the accept request. If $proposer$ receives accept responses from the majority of $acceptor$, it can confirm the consensus.
    
    \subsection{The proof the theorem \ref{theorem:1}}
    
    The theorem \ref{theorem:1} can be proved by induction.
    
    \begin{lemma}[induction step]
    Let \{$m$, $u$\} be the first proposal accepted by majority of $acceptor$.
    Let \{$n$, $v$\} be an issued proposal with $m < n$.
    Suppose that, for all $k \in [m, n)$, the proposal with $k$ has value $u$, then $v = u$
    \end{lemma}
    
    Let $M_1$ be the set of $acceptor$ that accepted \{$m$, $u$\}, proposal \{$n$, $v$\} is issued by some $proposer$ $p$ hence its prepare request must be replied from some majority set of $acceptor$, namely $M_2$.
    
    Let an $acceptor$ $q \in M_1 \cap M_2$, since $m < n$, from step 1 of \ref{algorithm:acceptor:prepare}, we know that $acceptor$ $q$ must accept \{$m$, $u$\} before receiving prepare request of $n$.
    
    Furthermore, $acceptor$ $q$ returns to the prepare request with the proposal \{$k$, $u^*$\} where it is the latest proposal $acceptor$ $q$ accepts.
    
    We can establish the relation among $m$, $k$, and $n$: $m \leq k < n$, by the induction step assumption, we can conclude that $u^* = u$
    
    From the perspective of $proposal$ $p$, it receives responses from $M_2$ where there is at least one non-$Null$ proposal. The maximum $proposal\_number$ from the responses set is greater than or equal to $k$, one of the responses, namely $k^*$.
    
    We can further establish the relation between $k^*$ and $n$. Since, $proposer$ $p$ receives $k^*$ from an $acceptor$ that accepted a proposal with $proposal\_number$ $k^*$, from step 1 of \ref{algorithm:acceptor:prepare}, $k^* < n$
    
    Therefore, $m \leq k \leq k^* < n$, that implies $proposer$ $p$ picks the $value$ corresponding to the highest $proposer\_number$ $k^*$ which is $u$. Or, $v = u$


\bibliography{references}

\end{document}
